\begin{document}
\begin{rawproblem}{input.txt}{output.txt}

Найти вершины, через которые проходит нечётное число наибольших полупутей, и
удалить (правым удалением) ту из них, ключ которой наибольший.
Если в дереве нет вершин, удовлетворяющих нужному свойству, то ничего
удалять не требуется. Выполнить прямой (левый) обход полученного дерева.

\InputFile

Входной файл содержит последовательность чисел~--- ключи вершин в порядке добавления в дерево. Гарантируется, что в дереве не менее двух вершин.

\OutputFile

Выходной файл должен содержать последовательность ключей вершин, полученную прямым левым обходом итогового дерева.

\Example

% this is a comment
\% this is not a comment

\begin{verbatim}
1
2 % Comments are visible in verbatim
3
4
5
\end{verbatim}

\begin{example}%
\exmp{% input
10
5
20
4
6
15
30
3
7
14
40
8
50
9
60
}{% output
10
5
4
3
6
7
8
9
20
15
14
30
40
50
}%
\end{example}%
\begin{center}%
    \includegraphics{tst01-1.GIF}%
\end{center}%

\end{rawproblem}
\end{document}
